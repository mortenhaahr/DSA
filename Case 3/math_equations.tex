
\documentclass[a4paper,onecolumn,oneside]{report}
%Figurtekst ændret udseende
\usepackage[font=small,labelfont=bf,listformat=simple]{caption}
% Tillader H som float specifier
\usepackage{float}
% Til figurer
\usepackage{graphicx}
%Figurtekst ændret udseende
\usepackage[font=small,labelfont=bf,listformat=simple]{caption}

% Lange ligninger
\usepackage{amsmath}

% cdot automatisk når man skriver *
\mathcode`\*="8000
{\catcode`\*\active\gdef*{\cdot}}

%Edit Kapitler og sectioner
\usepackage{tocloft}
\usepackage{titlesec}
    \setcounter{tocdepth}{5}
    \setcounter{secnumdepth}{5}

%Udseende på overskrifter
\titleformat{\chapter}
    {\Huge \bfseries}                          % format af tallet
    {\thechapter} % \thechapter                % label 
    {20pt}                                     % indhak mellem tal og skrift
    {\Huge \bfseries \vspace{0pt}}             % skriftstørrelse og type

\titleformat{\section}
	{\Large \bfseries} 	                       % format af tallet
	{\thesection}	                           % label
	{5pt}             	                       % indhak mellem tal og skrift
	{\Large \vspace{0pt}}			           % skriftstørrelse og type

\titleformat{\subsection}
	{\large \bfseries}                         % format af tallet
	{\thesubsection}                           % label
	{5pt}             			               % indhak mellem tal og skrift
	{\large \vspace{0pt}}           		   % skriftstørrelse og type
	
\titleformat{\subsubsection}
	{\normalfont \bfseries}                    % format
	{}                                         % label
	{0pt}             			               % indhak mellem tal og skrift
	{\normalfont \bfseries \vspace{0pt}}       % skriftstørrelse og type


%Fjerne start indhakket i alle textbokse.
\parindent = 0pt
\parskip = 1em

\begin{document}

\chapter{Opgave 2.2}
Opgaven går ud på at finde systemfunktionen H(z) og opskriv differensligningen herudfra.

Der tages udgangspunkt i et standard andenordens filter i z-domænet med konjugeret kompleks polpar:
\begin{equation}
    H(z)= \frac{(z-z_0)(z-\bar{z_0})}{(z-p_0)(z-\bar{p_0})}
\end{equation}

Parenteserne udvides ved at gange ind i dem:
\begin{equation}
    H(z)=\frac{z^2 - z*\bar{z_0} - z_0*z + z_0*\bar{z_0}}{z^2 - z*\bar{p_0} - p_0*z + p_0*\bar{p_0}} 
\end{equation}

Udtrykkene med $z_0*\bar{z_0}$ og $p_0*\bar{p_0}$ kan omskrives, da de er kompleks konjugeret, og den distributive regel anvendes:
\begin{equation}
    H(z)=\frac{z^2 - z*(z_0+\bar{z_0}) + |{z_0}^2|}{z^2 - z*(p_0 +\bar{p_0}) + |{p_0}^2|} 
\end{equation}

Ganger $z^{-2}$ på tæller og nævner:
\begin{equation}
    H(z)= \frac{1 - z^{-1}*(z_0+\bar{z_0}) + z^{-2}*|{z_0}^2|}{1 - z^{-1}*(p_0 +\bar{p_0}) + z^{-2}*|{p_0}^2|} 
\end{equation}

Omskrives til forhold mellem $Y(z)$ og $X(z)$:
\begin{equation}
    H(z)= \frac{Y(z)}{X(z)} = G * \frac{1 - z^{-1}*(z_0+\bar{z_0}) + z^{-2}*|{z_0}^2|}{1 - z^{-1}*(p_0 +\bar{p_0}) + z^{-2}*|{p_0}^2|} 
\end{equation}

Ganger X(z) på begge sider:
\begin{equation}
    Y(z) = X(z)  * \frac{1 - z^{-1}*(z_0+\bar{z_0}) + z^{-2}*|{z_0}^2|}{1 - z^{-1}*(p_0 +\bar{p_0}) + z^{-2}*|{p_0}^2|}
\end{equation}

Ganger nævneren på begge sider og udvider venstresiden:
\begin{equation}
    \begin{split}
        Y(z) - Y(z)* z^{-1}*(p_0 +\bar{p_0}) + Y(z)*z^{-2}*|{p_0}^2| \\
        = X(z) * (1 - z^{-1}*(z_0+\bar{z_0}) + z^{-2}*|{z_0}^2|)        
    \end{split}
\end{equation}

Flytter de fleste led fra venstresiden. over på højresiden:
\begin{equation}
    \begin{split}
    Y(z) = X(z) * (1 - z^{-1}*(z_0+\bar{z_0}) + z^{-2}*|{z_0}^2|) \\
    + Y(z)* z^{-1}*(p_0 +\bar{p_0}) - Y(z)*z^{-2}*|{p_0}^2|
    \end{split}
\end{equation}

Udvider X leddet:
\begin{equation}
    \begin{split}
    Y(z) = X(z) - X(z) *z^{-1}*(z_0+\bar{z_0}) + X(z) * z^{-2}*|{z_0}^2|\\ + Y(z)* z^{-1}*(p_0 +\bar{p_0}) - Y(z)*z^{-2}*|{p_0}^2|
    \end{split}
\end{equation}

Transformerer til diskrete tidsdomæne (differensligning):\footnote{Sætter X(z) og Y(z) = x(n) og y(n), samt $z^{-1}$ svarer til en forsinkelse på 1 sample.}
\begin{equation}
    \begin{split}
    y(n) = x(n) - x(n-1) *(z_0+\bar{z_0}) \\
    + x(n-2)*|{z_0}^2| + y(n-1)*(p_0 +\bar{p_0}) - y(n-2)*|{p_0}^2|
    \end{split}
\end{equation}


Bruger komplekse regneregler til at simplificere $(z_0 +\bar{z_0})$ og $(p_0 +\bar{p_0})$:
\begin{equation}
    \begin{split}
    y(n) = x(n) - x(n-1) *(2*\textrm{Re}(z_0)) + x(n-2)*|{z_0}^2| \\
     + y(n-1)*(2*\textrm{Re}(p_0)) - y(n-2)*|{p_0}^2|
    \end{split}
\end{equation}

Nu kan filterkoefficienter identificeres:
\begin{equation}
    \begin{split}
        z_0  = \exp(0.1028*1j) = \cos(0.1028) + j*\sin(0.1028) = 0.9947 \pm 0.1026j \\
        p_0 = 0.99*\exp(0.1028*1j) = 0.99*\cos(0.1028) + 0.99j*\sin(0.1028) = 0.9848 \pm 0.1016j\\
        b_0 = 1\\
        b_1 = -2*\textrm{Re}(z_0) = -2*\cos(0.1028) = -1.9894\\
        b_2 = |{z_0}^2| = |(\cos(0.1028)+j*\sin(0.1028))^2| = 1 \\
        a_0 = 1\\
        a_1 = (-1)*2*\textrm{Re}(p_0) = -2*0.99*\cos(0.1028) = -1.9695\\
        a_2 = (-1)* -|{p_0}^2| = |(0.99*\exp(0.1028*1j)^2)| = 0.9801
    \end{split}
    \label{eq:opgave2_f:done}
\end{equation}

\textbf{Bemærk:} at der ved $a_1$ og $a_2$ ganges -1 på. Dette skyldes at vi aflæser vores filterkoefficienter gennem differensligningen, mens de er på højresiden, men de skal aflæses fra venstresiden for at det passer.\\
Hvis ikke vi havde (-1) foran, ville $a_0$ blive aflæst fra venstresiden og $a_1$ og $a_2$ fra højresiden, så nogle af vores koefficienter ville have omvendt fortegn.

% Note: Koefficienterne stemmer overens med det vi får i Matlab

\bibliographystyle{plain}
\bibliography{references.bib}
\end{document}